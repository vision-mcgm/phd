% Updated in March 2012 by Yasuyuki Matsushita
% Updated in April 2002 by Antje Endemann, ...., and in March 2010 by Reinhard Klette
% Based on CVPR 07 and LNCS style, with modifications by DAF, AZ and elle 2008, AA 2010, ACCV 2010

\documentclass[runningheads]{llncs}
\usepackage{graphicx}
\usepackage{amsmath,amssymb} % define this before the line numbering.
\usepackage{ruler}
\usepackage{color}

%===========================================================
\begin{document}
\pagestyle{headings}
\mainmatter

\def\ACCV12SubNumber{***}  % Insert your submission number here

%===========================================================
\title{Author Guidelines for ACCV Submission} % Replace with your title
\titlerunning{ACCV-12 submission ID \ACCV12SubNumber}
\authorrunning{ACCV-12 submission ID \ACCV12SubNumber}

\author{Anonymous ACCV 2012 submission}
\institute{Paper ID \ACCV12SubNumber}

\maketitle

%===========================================================
\begin{abstract}
The abstract should summarize the contents of the paper and should
contain at least 70 and at most 300 words. It should be set in 9-point
font size and should be inset 1.0~cm from the right and left margins.
\dots
\end{abstract}

%===========================================================
\section{Introduction}

Please follow the steps outlined below when submitting your manuscript.
%Do not use any additional Latex macros.

%------------------------------------------------------------------------- 
\subsection{Language}

All manuscripts must be in English.
%; uniformly in British English is preferred.

%------------------------------------------------------------------------- 
\subsection{Paper length}
The basic length is 12 pages, but up to two additional pages may be
purchased in the final electronic and printed proceedings. This brings the {\em maximum} length for submission to 14 pages. Over-length papers will
simply not be reviewed. This includes papers where the margins and
formatting are deemed to have been significantly altered from those
laid down by this style guide. The reason such papers will not be
reviewed is that there is no provision for supervised revisions of
manuscripts. 

%------------------------------------------------------------------------- 
\subsection{Dual submission}

By submitting a manuscript to ACCV, the author(s) assert that it has
not been previously published in substantially similar
form. Furthermore, no paper which contains significant overlap with
the contributions of this paper either has been or will be submitted
during the ACCV 2012 review period to either a journal or a
conference. However, the manuscript may also be submitted  to
one workshop that is accompanying ACCV 2012 in Daejeon (see Fig.~\ref{fig:accv12}).

\begin{figure}
\centering
\includegraphics[width=120mm]{accv12}
\caption{
The website of ACCV 2012 is at \url{http://www.accv2012.org}. 
%Accompanying workshops are listed on \url{http://www.accv2012.org/workshops.html}. 
If images are copied from some source then provide the reference.
Follow copyright rules as they apply. A caption ends with a full stop.}
\label{fig:accv12}
\end{figure}

If there are any papers that may appear to the reviewers to violate
this condition, then it is your responsibility to (1) cite these
papers (preserving anonymity as described in Section~\ref{sec:blind}
of this example paper, (2) argue in the body of your paper why your
ACCV paper is non-trivially different from these concurrent
submissions, and (3) include anonymized versions of those papers in
the supplemental material.

%------------------------------------------------------------------------- 
\subsection{Supplemental Material} 

Authors may optionally upload supplemental material. Typically, this
material might include videos of results that cannot be included in
the main paper, anonymized related submissions to other conferences
and journals, and appendices or technical reports containing extended
proofs and mathematical derivations that are not essential for
understanding of the paper. Note that the contents of the supplemental
material should be referred to appropriately in the paper and that
reviewers are not obliged to look at it.

All supplemental material must be zipped or tarred into a single file. There is a 30~MB limit on the size of this file. The deadline for supplemental material is a week after the main paper deadline.

%------------------------------------------------------------------------- 
\subsection{Line numbering}

All lines should be numbered, as in this example document. This makes
reviewing more efficient, because reviewers can refer to a line on a
page. If you are preparing a document using a non-\LaTeX\
document preparation system, please arrange for an equivalent line numbering. Note that accepted papers need to be submitted as a \LaTeX\
document in the style as defined in this document.

%------------------------------------------------------------------------- 
\subsection{Mathematics}

Please number all of your sections and displayed equations.  Again,
this makes reviewing more efficient, because reviewers can refer to a
line on a page.  Also, it is important for readers to be able to refer
to any particular equation.  Just because you did not refer to it in
the text does not mean some future reader might not need to refer to
it.  It is cumbersome to have to use circumlocutions like ``the
equation second from the top of page 3 column 1.''  (Note that the
line numbering will not be present in the final copy, so is not an
alternative to equation numbers).  Some authors might benefit from
reading Mermin's description of how to write mathematics:
\url{http://www.cvpr.org/doc/mermin.pdf}.

%===========================================================
\section{Blind review}
\label{sec:blind}

Many authors misunderstand the concept of anonymizing for blind
review.  Blind review does not mean that one must remove
citations to one's own work---in fact it is often impossible to
review a paper unless the previous citations are known and
available.

Blind review means that you do not use the words ``my'' or ``our''
when citing previous work.  That is all.  (But see below for 
techreports).

Saying ``this builds on the work of Lucy Smith [1]'' does not say
that you are Lucy Smith, it says that you are building on her
work.  If you are Smith and Jones, do not say ``as we show in
[7],'' say ``as Smith and Jones show in [7]'' and at the end of the
paper, include Reference~7 as you would any other cited work.

An example of a bad paper:
\begin{quote}
\begin{center}
    {\bf An Analysis of the Frobnicatable Foo Filter}
\end{center}
   
   In this paper we present a performance analysis of our
   previous paper [1], and show it to be inferior to all
   previously known methods.  Why the previous paper was
   accepted without this analysis is beyond me.
   
   [1] Removed for blind review
\end{quote}
   
An example of an excellent paper:   
   
\begin{quote}
\begin{center}
     {\bf An Analysis of the Frobnicatable Foo Filter}
\end{center}
   
   In this paper we present a performance analysis of the
   paper of Smith and Jones [1], and show it to be inferior to
   all previously known methods.  Why the previous paper
   was accepted without this analysis is beyond me.
   
   [1] Smith, L., Jones, C.: The frobnicatable foo
   filter, a fundamental contribution to human knowledge.
   Nature {\bf 381} (2005) 1--213
\end{quote}
   
If you are making a submission to another conference at the same time,
which covers similar or overlapping material, you may need to refer to that
submission in order to explain the differences, just as you would if you
had previously published related work.  In such cases, include the
anonymized parallel submission~\cite{Authors12} as additional material and
cite it as
\begin{quote}
[1]  Authors: The frobnicatable foo filter, ACCV 2012 Submission ID 324,
Supplied as additional material {\tt bmvc12.pdf}.
\end{quote}

Finally, you may feel you need to tell the reader that more details can be found elsewhere, and refer them to a technical report.  For conference submissions, the paper must stand on its own, and not {\em require} the reviewer to go to a techreport for further details.  Thus, you may say in the body of the paper ``further details may be found in~\cite{Authors12b}.''  Then submit the anonymized techreport as additional material. Again, you may not assume the reviewers will read this material.

Sometimes your paper is about a problem which you tested using a tool which
is widely known to be restricted to a single institution.  For example,
let us say it is 1969, you have solved a key problem on the Apollo lander,
and you believe that the ACCV audience would like to hear about your
solution.  The work is a development of your celebrated 1968 paper entitled
``Zero-g frobnication: How being the only people in the world with access to
the Apollo lander source code makes us a wow at parties,'' by Zeus.

You can handle this paper like any other.  Do not write ``We show how to
improve our previous work [Anonymous, 1968].  This time we tested the
algorithm on a lunar lander [name of lander removed for blind review].''
That would be silly, and would immediately identify the authors. Instead
write the following:
%
\begin{quotation}
\noindent
   We describe a system for zero-g frobnication.  This
   system is new because it handles the following cases:
   A, B.  Previous systems [Zeus et al. 1968] did not
   handle case B properly.  Ours handles it by including
   a foo term in the bar integral.

   The proposed system was integrated with the Apollo
   lunar lander, and went all the way to the moon.  
\end{quotation}
%
As you can see, the above text follows standard scientific convention,
reads better than the first version, and does not explicitly name you as
the authors.  A reviewer might think it likely that the new paper was
written by Zeus, but cannot make any decision based on that guess.
He or she would have to be sure that no other authors could have been
contracted to solve problem B. \\

\noindent FAQ: Are acknowledgements OK?  -- Answer: No. Please {\it omit
acknowledgements} in your review copy; they can go in the final copy.

%===========================================================
\section{Manuscript Preparation}

This is an edited version of Springer LNCS instructions adapted for
ACCV 2012 full paper paper submission. 

You will have to use \LaTeX2$_\varepsilon$ for the
preparation of your final (accepted)
camera-ready manuscript together with the corresponding Springer
class file \verb+llncs.cls+.

We would like to stress that the class/style files and the template
should not be manipulated and that the guidelines regarding font sizes
and format should be adhered to. This is to ensure that the end product
is as homogeneous as possible.

%------------------------------------------------------------------------- 
\subsection{Printing Area}

The printing area is $122  \; \mbox{mm} \times 193 \;
\mbox{mm}$.
The text should be justified to occupy the full line width,
so that the right margin is not ragged, with words hyphenated as
appropriate. Please fill pages so that the length of the text
is no less than 180~mm.

%------------------------------------------------------------------------- 
\subsection{Layout, Typeface, Font Sizes, and Numbering}

Use 10-point type for the name(s) of the author(s) and 9-point type for
the address(es) and the abstract. For the main text, use 10-point
type and single-line spacing.
We recommend using Computer Modern Roman (CM) fonts, Times, or one
of the similar typefaces widely used in photo-typesetting.
(In these typefaces the letters have serifs, {\it i.e.}, short endstrokes at
the head and the foot of letters.)
Italic type may be used to emphasize words in running text. 

{\it Bold type and underlining should be avoided.}

With these sizes, the interline distance should be set so that some 45
lines occur on a full-text page.

%------------------------------------------------------------------------- 
\subsubsection{Headings.}

Headings should be capitalised
({\it i.e.}, nouns, verbs, and all other words
except articles, prepositions, and conjunctions should be set with an
initial capital) and should,
with the exception of the title, be aligned to the left.
Words joined by a hyphen are subject to a special rule. If the first
word can stand alone, the second word should be capitalised.
The font sizes
are given in Table~\ref{table:headings}. (Note that vertical lines
are not common table components anymore.)
%
\setlength{\tabcolsep}{4pt}
\begin{table}
\begin{center}
\caption{
Font sizes of headings. Table captions should always be
positioned {\it above} the tables. A table
caption ends with a full stop.
}
\label{table:headings}
\begin{tabular}{lll}
\hline\noalign{\smallskip}
Heading level $\qquad\qquad$& Example & Font size and style\\
\noalign{\smallskip}
\hline
\noalign{\smallskip}
Title (centered)  & {\Large \bf Lecture Notes \dots} $\qquad$& 14 point, bold\\
1st-level heading & {\large \bf 1 Introduction} & 12 point, bold\\
2nd-level heading & {\bf 2.1 Printing Area} & 10 point, bold\\
3rd-level heading & {\bf Headings.} Text follows \dots & 10 point, bold
\\
4th-level heading & {\it Remark.} Text follows \dots & 10 point,
italic\\
\hline
\end{tabular}
\end{center}
\end{table}
\setlength{\tabcolsep}{1.4pt}

Here are
some examples of headings: ``Criteria to Disprove Context-Freeness of
Collage Languages,'' ``On Correcting the Intrusion of Tracing
Non-deterministic Programs by Software,'' ``A User-Friendly and
Extendable Data Distribution System,'' ``Multi-flip Networks:
Parallelizing GenSAT,'' ``Self-determinations of Man.''

%------------------------------------------------------------------------- 
\subsubsection{Lemmas, Propositions, and Theorems.}

The numbers accorded to lemmas, propositions, theorems, and so forth should
appear in consecutive order, starting with the number one, and not, for
example, with the number eleven.

%------------------------------------------------------------------------- 
\subsection{Figures and Photographs}
\label{sect:figures}

Produce your figures electronically and integrate
them into your text file. We recommend using package
\verb+graphicx+ or the style files \verb+psfig+ or \verb+epsf+.

Check that in line drawings, lines are not
interrupted and have constant width. Grids and details within the
figures must be clearly readable and may not be written one on top of
the other. Line drawings should have a resolution of at least 800 dpi
(preferably 1200 dpi).
For digital halftones 300 dpi is usually sufficient. Color is possible in
figures, but note that figures in the printed proceedings will be in 
halftones only.

The lettering in figures should have a height of 2~mm (10-point type).
Figure~\ref{fig:accv12} contains lettering of different sizes; in such a case make sure that the smallest letters have a hight of 2~mm.
Figures should be scaled up or down accordingly.
Do not use any absolute coordinates in figures.

Figures should be numbered and should have a caption which should
always be positioned {\it under} the figures, in contrast to the caption
belonging to a table, which should always appear {\it above} the table.
Please center the captions between the margins and set them in
9-point type (Figs.~\ref{fig:accv12} and \ref{fig:example} show examples).
The distance between text and figure should be about 8~mm, the
distance between figure and caption about 5~mm.
%
\begin{figure}
\centering
\includegraphics[height=72mm]{eijkel2}
\caption{
One kernel at $x_s$ ({\it dotted kernel}) or two kernels at
$x_i$ and $x_j$ ({\it left and right}) lead to the same summed estimate
at $x_s$. This shows a figure consisting of different types of
lines. Elements of the figure described in the caption should be set in
Italics and in parentheses, as shown in this sample caption.
}
\label{fig:example}
\end{figure}

If possible define figures as floating
objects, or use location parameters ``t'' or ``b'' for ``top'' or ``bottom.'' Avoid using the location
parameter ``h'' for ``here.'' If you have to insert a page break before a
figure, ensure that the previous page is completely filled.

%------------------------------------------------------------------------- 
\subsection{Formulas}

Displayed equations or formulas are centered and set on a separate
line (with an extra line or halfline space above and below). Displayed
expressions should be numbered for reference. The numbers should be
consecutive within each section or within the contribution,
with numbers enclosed in parentheses and set on the right margin.
For example,
%
\begin{equation}
  \psi (u) = \int_{o}^{T} \left[\frac{1}{2}
  \left(\Lambda_{o}^{-1} u,u\right) + N^{\ast} (-u)\right] \mathrm{d}t~.
  \label{equ:dt}
\end{equation}

Please punctuate a displayed equation in the same way as ordinary
text but with a small space before the end punctuation.

%Do not punctuate a displayed equation in the same way as ordinary
%text; for example, there is no full stop at the end of Equ.~(\ref{equ:dt}).

%------------------------------------------------------------------------- 
\subsection{Program Code}

Program listings or program commands in the text are normally set in
typewriter font, for example, CMTT10 or Courier.

\medskip

\noindent
{\it Example of a Computer Program}
\begin{verbatim}
program Inflation (Output)
  {Assuming annual inflation rates of 7%, 8%, and 10%,...
   years};
   const
     MaxYears = 10;
   var
     Year: 0..MaxYears;
     Factor1, Factor2, Factor3: Real;
   begin
     Year := 0;
     Factor1 := 1.0; Factor2 := 1.0; Factor3 := 1.0;
     WriteLn('Year  7% 8% 10%'); WriteLn;
     repeat
       Year := Year + 1;
       Factor1 := Factor1 * 1.07;
       Factor2 := Factor2 * 1.08;
       Factor3 := Factor3 * 1.10;
       WriteLn(Year:5,Factor1:7:3,Factor2:7:3,Factor3:7:3)
     until Year = MaxYears
end.
\end{verbatim}
%
\noindent
{\small (Example from Jensen K., Wirth N. (1991) Pascal user manual and
report. Springer, New York)}

%------------------------------------------------------------------------- 
\subsection{Footnotes}

The superscript numeral used to refer to a footnote appears in the text
either directly after the word to be discussed or -- in relation to a
phrase or a sentence -- following the punctuation sign (comma,
semicolon, or full stop). Footnotes should appear at the bottom of
the
normal text area, with a line of about 2~cm 
immediately above them.\footnote
{
   The footnote numeral is set flush left
   and the text follows with the usual word spacing. Second and subsequent
   lines are indented. Footnotes should end with a full stop.
}

%------------------------------------------------------------------------- 
\subsection{Citations}

The list of references is headed ``References'' and is not assigned a
number in the decimal system of headings. The list should be set in small print and placed at the end of your contribution, in front of the appendix, if one exists.

Do not insert a page break before the list of references if the page is not completely filled. Citations in the text are with square brackets and consecutive numbers, such as \cite{Alpher02}, or \cite{Alpher03,Herman04}.

%References are listed in alphabetic order by the surname of the first author, or the identifying word (e.g., in case of a website). Have
%all anonymized references at the beginning of the list.

%here would be your acknowledgement (if any) in the final accepted paper

%===========================================================
\bibliographystyle{splncs}
\bibliography{egbib}

%this would normally be the end of your paper, but you may also have an appendix
%within the given limit of number of pages
%\end{document}

%===========================================================
\clearpage\mbox{}Page \thepage\ of the manuscript.
\clearpage\mbox{}Page \thepage\ of the manuscript.
\clearpage\mbox{}Page \thepage\ of the manuscript.
\clearpage\mbox{}Page \thepage\ of the manuscript.
12 pages are without extra charge for additional pages.
\par\vfill\par
This is the minimum number of pages required for ACCV 2012.

\clearpage\mbox{}Page \thepage\ of the manuscript.
\clearpage\mbox{}Page \thepage\ of the manuscript.
Pages 13 and 14 are ``additional pages''; they will need extra payment when accepted.
\par\vfill\par
Now we have reached the maximum size of the ACCV 2012 submission.

\end{document}
