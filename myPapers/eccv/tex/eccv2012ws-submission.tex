% last updated in April 2002 by Antje Endemann
% Based on CVPR 07 and LNCS, with modifications by DAF, AZ and elle, 2008 and AA, 2010, and CC, 2011

\documentclass[runningheads]{llncs}
\usepackage{graphicx}
\usepackage{amsmath,amssymb} % define this before the line numbering.
\usepackage{ruler}
\usepackage{color}
\usepackage{subfigure}
\usepackage[width=122mm,left=12mm,paperwidth=146mm,height=193mm,top=12mm,paperheight=217mm]{geometry}
\begin{document}
% \renewcommand\thelinenumber{\color[rgb]{0.2,0.5,0.8}\normalfont\sffamily\scriptsize\arabic{linenumber}\color[rgb]{0,0,0}}
% \renewcommand\makeLineNumber {\hss\thelinenumber\ \hspace{6mm} \rlap{\hskip\textwidth\ \hspace{6.5mm}\thelinenumber}}
% \linenumbers
\pagestyle{headings}
\mainmatter
\def\ECCV12SubNumber{295}  % Insert your submission number here

\title{Mimicry Evaluation and Identity Blending with Morph Space PCA} % Replace with your title

\titlerunning{ECCV-12 submission ID \ECCV12SubNumber}

\authorrunning{ECCV-12 submission ID \ECCV12SubNumber}

\author{Anonymous ECCV submission}
\institute{Paper ID \ECCV12SubNumber}


\maketitle

\begin{abstract}
The skill of expression mimicry must be learned without direct performance feedback\cite{grossberg2010children}, and study of this phenomenon would benefit from a metric characterising quality of imitation. We describe a face modelling tool allowing image representation in a high-dimensional morph space, compression to a small number of coefficients using PCA\cite{jolliffe2002principal}, and expression transfer between face models by projection of the source morph description into the target morph space. This technique allows statistical evaluation of mimicry quality and creation of an identity-blended avatar model whose high degree of realism suggests diverse applications in psychophysics, animation and affective computing.
\end{abstract}


\section{Introduction}


\bibliographystyle{splncs}
\bibliography{egbib}





\end{document}
